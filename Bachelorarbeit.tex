% ----------------------------------------------------
% Magic Comments for LaTeX-editors (TeXstudio, VSCode with LaTeX Workshop extension...)
% !TeX encoding = utf8
% !TeX spellcheck = de_DE
% !TeX program = pdflatex
% !BIB program = biber
% ----------------------------------------------------

% ----------------------------------------------------
% Load the HfTL-Thesis class
\documentclass[
% Select the main Language of your thesis here. Titlepage, captions and statement of authorship will change accordingly:
%
ngerman,% - Deutsch (default)
% USenglish % - American english
% UKenglish % - British english
%
numeric % (default) use numeric citation style sorted by the occurence in text: [5]
% alphabetic % - use alphabetic citation style: [Lau95]
% authoryear % - use authoryear citation style: Laubach 1995
]{Bachelorarbeit}
% ----------------------------------------------------

% ----------------------------------------------------
% Set your custom options here. See https://komascript.de/~mkohm/scrguide.pdf for possibilities.

\KOMAoption{twoside}{true} % Enable double sided layout
\KOMAoption{BCOR}{8.25mm} % Binding Correction, adds the specified margin to the inner side of the page (left on single sided layout). This is to compensate lost page margin due to binding.

% ----------------------------------------------------

% ----------------------------------------------------
% Load your own packages here

\usepackage{blindtext} % Only for demonstration purposes. Can be removed!
\usepackage{tikz, pgfplots} % To create diagrams and figures directly with LaTeX
\usepackage{listings} % support for Sourcecodelistings. For more advanced, prettier highlighting use the minted package instead (https://github.com/gpoore/minted) -> Requires Python with Pygments installed.
\usepackage{siunitx} % provides support for SI unit handling
\usepackage[useregional]{datetime2} % support for formatted date
\usepackage{tocbasic} % support indent in table of contents
\usepackage{graphicx}
\usepackage{bera}% optional: just to have a nice mono-spaced font
\usepackage{xcolor}
\usepackage{animate}
\usepackage{pdfpages}

\setkomafont{caption}{\small}

% Options for table of contents
\renewcommand*{\tableofcontents}{\listoftoc[\contentsname]{toc}}
\DeclareTOCStyleEntries[raggedentrytext]{tocline}{section,subsection,subsubsection,paragraph,subparagraph}
 

% ----------------------------------------------------

% ----------------------------------------------------
% Variables for Titlepage, PDF properties...
% Below are the default values. Comment them out to change them!

\university{Hochschule für Telekommunikation Leipzig (FH)}
\institute{Institut für Telekommunikationsinformatik}
\documentKind{Abschlussarbeit zur Erlangung des akademischen Grades}
\forDegree{Bachelor of Science}
%
\topic{\enquote{Inwiefern bietet die Authentifikation ohne Passwort Vor- und Nachteile gegenüber Webanwendungen hinsichtlich Nutzerfreundlichkeit, Datenschutz und Sicherheit?}} 
%
\thesisAuthor{Ertugrul Sener}
\authorDateOfBirth{17.10.1998}
\authorPlaceOfBirth{Berlin}
%
\handInDate{\today} %formatted custom date
%
\firstReviewerName{Prof. Dr. Erik Buchmann}
\firstReviewerInstitution{Hochschule für Telekommunikation Leipzig}
\firstReviewerInstitutionStreet{Gustav-Freytag-Straße 43-45}
\firstReviewerInstitutionPlace{04277 Leipzig}
%
\secondReviewerName{Juri Lobov}
\secondReviewerInstitution{T-Systems International GmbH}
\secondReviewerInstitutionStreet{Holzhauser Straße 1-4}
\secondReviewerInstitutionPlace{13509 Berlin}

% ----------------------------------------------------

% ----------------------------------------------------
% Include own references
\addbibresource{References/esener.bib}
% ----------------------------------------------------

% ----------------------------------------------------
% Set a list of directories, where graphics are stored, so you dont have to prepend the subdirectory in \includegraphics. The "Graphics" folder is set by default.

%\graphicspath{{./Graphics/}, {./Images}}
% ----------------------------------------------------

% ----------------------------------------------------
% Load acronym definitions
\input{acronyms}
\makenoidxglossaries % index acronyms
% ----------------------------------------------------

% ----------------------------------------------------
% The document body. Start your work here.
\begin{document}
% \frontmatter % page numbers as small roman letters
\maketitle % print the titlepage
\cleardoublepage % start a new page

\chapter{Vorwort}
Hier is

\blindtext
 % include the file Content/vorwort.tex
\cleardoublepage

\tableofcontents %print the table of contents
\cleardoublepage

% Include the other chapters
% !TeX root = ../my-thesis.tex
\chapter{Einleitung}

\section{Motivation}
Die Motivation für die Arbeit erschloss sich mir am 29. Juni 2020, als eine Rundmail von dem Monitoring-Tool von Firefox herumging, welches einen neuen Datenleck bei Wattpad meldete. Wattpad ist eine Webseite, bei welcher Nutzer Geschichten frei für die Öffentlichkeit schreiben und publizieren können. Kompromittiert wurden neben Passwörtern auch IP-Adressen, E-Mail-Adressen, Geburtsdaten und sehr persönliche Informationen wie geografische Standorte, ein Kurzprofil, das Profilbild, die Social-Media-Profile und alle verknüpften Accounts jeglicher Dienste des Benutzers. Die Datenmenge umschloss etwa 300 Millionen Einträge, bei denen die Passwörter zwar gesalted waren, allerdings das Knacken dieser bereits durch eine Hackercommunity in Angriff genommen und etwa 10\% bereits 'entschlüsselt' wurden. Die Frage die sich stellte war es, ob persönliche Daten von mir nun im Umlauf waren. Glücklicherweise nutze ich einen Passwort-Manager und mein Passwort war zufallsgeneriert und ich kannte es selbst nicht einmal, da es zu kompliziert war um es sich zu merken. Dieses Leck hatte sich in meinem Freundeskreis, unter denen einige auf Wattpad registriert waren, rumgesprochen und mir stellte sich immer mehr eine Kernfrage: Wie kann es im Jahre 2020 immernoch sein, dass Menschen die selben Passwörter für Dienste verwenden und diese nicht ausreichend kompliziert wählen? Immerhin gibt es dieses Problem nun schon seit einigen Jahren und dennoch ändert sich augenscheinlich nichts an der Situation. Technische Probleme und Angriffsszenarien kriegt die informationsafine Menscheit gut organisiert, doch dann scheitert es in riesigen Unternehmen an der simplen Wahl eines Passwortes. Bei mehreren Disskusionen in meinem Umfeld, kamen die selben Argumente gegen eine zwei Faktor Authentifizierung und Passwort-Manager hervor. Passwörter seien unmöglich für alle Dienste merkbar und eine zwei Faktor Authentifizierung sei zu kompliziert um sie für jeden Dienst, wenn möglich, einzusetzen. Die Forschungsfrage kam daher, dass ich mich fragte, ob man mit einer Kombination aus den vorhandenen vielfältigen Authentifizierungsmöglichkeiten eine beqeueme aber gleichzeitig auch sichere Authentifizierungsvariante schaffen könnte, bei denen es dem Anwender kinderleicht möglich sein soll sich gegenüber einer Webseite zu authentifizieren. Das Merken von langen Passwörtern sollte mitunter vermieden werden, da diese bei Kompromittierung publik werden und danach ein potenzielles Risiko für jeden anderen genutzten Dienst des Nutzers darstellen.

\section{Problemdefinition}
Lange reichte der Benutzername und das Passwort aus, um einen Benutzer sicher zu identifizieren. Heutzutage gelangen sensible Informationen immer öfter an Dritte. Dabei ist es nicht zwingend notwendig, dass die Authentifizierungsquelle, also jene Quelle bei der die Daten persistiert sind und die die Authentifizierung bei Eingabe durchführt, diese Daten irrtümlich herausgibt. Durch sogenannte Metadaten, das sind Informationen und Merkmale zu Daten, ist es Angreifern häufig möglich das Passwort zu erraten bzw. zurückzusetzen. Wenn also der Benutzername lautet 'MichaelJacksonFanForever' ist die Antwort auf die Sicherheitsfrage zum Lieblingskünstler nicht weit entfernt.
Gleichzeitig neigen Menschen zu leicht merkbaren Passwörtern, die sie mehrfach für verschiedene Dienste verwenden. Dies stützt eine kürzlich durchgeführte repräsentative Studie der Bitkom Reserach [1] im Auftrag des Digitalverbands Bitkom. So nutze etwa jeder dritte Onlinenutzer (36\%) dasselbe Passwort für mehrere Dienste. Auch wenn gleichzeitig 63\% der Befragten angaben, bei der Erstellung von Passwörtern auf ``einen Mix aus Buchstaben, Zahlen und Sonderzeichen'' zu achten, beweist diese Befragung an 1.000 Internetznutzern, dass die Frage nach der Sicherheit von Passwörtern auch im Jahre 2020 immernoch Relevanz hat. Eine ähnliche Studie hat die Bitkom zum Thema 'Nachlässigkeit bei Passwörtern' am 08.11.2016 [2] gemacht, bei der die prozentuale Verteilung an unsicheren Passwortnutzern die befragt wurden nur einen Prozent höher liegt. Das heißt konkret, dass sich innerhalb von 4 Jahren keine messbare Besserung ergeben hat. Das Bewusstsein über die Internetpräsenz und der Schutz dessen scheinen immernoch keine große Aufmerksamkeit vom modernen Nutzer zu erhalten. Das Problem mit unsicheren Passwörtern ist allerdings so alt wie das Internet.

\section{Stand der Forschung}
\subsection{Unsichere Passwörter}
In der fünften Ausgabe der Zeitschrift ``Wirtschaftsinformatik \& Management''  2018 mit dem Titel ``Schwache Passwörter - Nutzer spielen weiterhin Vogel Strauß'' schrieb der Autor Geralt Beuchelt: ``Der Umgang mit Passwörtern ist so ähnlich wie eine Diät: Eigentlich weiß man genau, was richtig ist - Macht aber oft genug das Gegenteil. Und nicht selten ist der Grund Bequemlichkeit. Warum selber kochen, wann nach einem langen Tag eine Pizza lockt? Und warum lange, umständliche Passwörter verwenden, wenn es einfach zu merkende, die man für alle Accounts verwendet, doch auch tun?``. [3]
Die symbolische Pizza steht für die Mehrfachverwendung von teils schwachen Passwörtern für alle genutzten Dienste inklusive des 'Verwaltungsdienstes' wie der Mail, welches als meist einziges Identifikationsmerkmal dient, über die weitere Dienste betroffen sein können. Der Begriff des Passwortes stammt aus dem militärischen Bereich des 16. Jahrhunderts, wobei tatsächlich das einzelne Wort gemeint war, welches einem Zutritt zu Gebäuden verschaffte. Damit verwandt ist das Kennwort, welches nicht das Passieren sondern die Kennung des gemeinsamen Geheimnisses betont. Damit ist gemeint, dass der Passierer mit einem Kennwort auch automatisch als Geheimnisträger identifiziert wird. Als allerdings Computer immer leistungsfähiger wurden, wurde der Begriff der Passphrase etabliert, um die Notwendigkeit längerer Passwörter hervorzuheben. Weitere Schlüsselwörter für das heutzutage bekannte Passwort sind: Schlüsselwort, Kodewort (auch: Codewort) oder die Parole. Die Länge gillt gemeinhin als die allumfassende Sicherheit von Passwörtern. Dem widerspricht die klare Trennung zwischen Länge und Komplexität von Passwörtern durch das Bundesamt für Sicherheit in der Informationstechnik, die sinngemäß in ihrer Empfehlung zum Thema 'sichere Passwörter' schreiben, dass die Länge von Passwörtern nicht dessen Komplexität und Sicherheit gegen Angriffe widerspiegelt [4].

\subsection{Passwort Policys}
Das gewählte Passwort soll für einen selbst leicht merkbar, für einen Computer oder menschlichen Angreifer allerdings schwer zu erraten sein. So empfielt das \ac{bsi} Passwörter zu verwenden, die möglichst nicht aus Tastaturmustern bestehen wie 'asdfgh' oder '1234abcd'. Allgemein wird ein 20 bis 25 Zeichen langes Passwort aus zwei Zeichenarten einem 8 bis 12 Zeichen langem Passswort aus 4 Zeichenarten in Punkto Komplexität gleichgesetzt. [4]\\
Das Wort 'Policy' ist in diesem Zusammenhang als 'die Regel' zu verstehen und in der Wortkombination sind Passwort Policys die Regeln, die zu einem sicheren Passwort führen. Derart Regeln gibt das \ac{bsi} vor. So seien der Kreativität bei Passwörtern keine Grenzen gesetzt [4]. Zum Beispiel könne man einen leicht zu merkenden Satz nehmen, diesen mit Bindestrichen verbinden und von jedem Wort den ersten Buchstaben entfernen. Die Frage die sich dabei stellt ist es, ob dieser Satz dann auch die Tippgeschwindigkeit des Nutzers beeinträchtigt, weil relativ viele Denkprozesse während des Tippens stattfinden müssen. Ein Passwort soll leicht merkbar sein, für einen Angreifer allerdings schwer zu brechen. Grundsätzlich gillt, auch das ist nur im Idealfall so, je länger ein Passwort ist, desto besser. Dies bedeutet wie bereits oben angeschnitten allerdings nicht, dass das Passwort 'aaaaaaaaaaaaaaaaaaaa' ein mathematisch sicheres Passwort ist. Die Länge des Passwortes ist nur einer von vielen Faktoren, die am Ende zur Komplexität und der daraus resultierenden beitragen. Im Idealfall besteht das lange Passwort aus mehreren Zeichenarten. Eine weitere Empfehlung ist es, keine Sonderzeichen an den Anfang oder das Ende des Passwortes anzuhängen, um es für einen Angreifer schwerer erratbar zu machen. Dies lässt sich damit begründen, dass sobald ein Angreifer die restlichen Zeichen des Passwortes erraten konnte oder durch Metadaten anderer Dienste (wie oben in dem Michael-Jackson Beispiel) kennt, das Durchprobieren von allen verfügbaren Sonderzeichen für die erste und letzte Stelle der Zeichenkette keine große Leistung erfordert. Sie machen das Passwort mathematisch zwar sicherer (Mehr Zeichenarten bedeuten mehr Zeichen insgesamt und dadurch mehr Kombinationsmöglichkeiten für Passwörter), bei gegebenen Umständen sind diese einzelnen Sonderzeichen allerdings obsolet und können weggelassen werden.
Passwort Policys können allerdings auch teilweise wertvolle Informationen für einen potenziellen Angreifer bieten. Denn was Angreifer durch sehr strikte Passwort Policys unteranderem erkennen können, ist die Mindest- und Maximalzeichenlänge. Dabei wird der Angreifer zum Beispiel aus der Regel 'Das Passwort muss mindestens 8 und maximal 16 Zeichen lang sein.' alle Kombinationen für weniger als 8 Zeichen und mehr als 16 Zeichen bei der Erratung eliminieren können. Weitere Regeln wie 'Das Passwort muss mindestens ein Sonderzeichen beinhalten' können zusätzliche Informationen bieten. Daher sind Passwort-Policys zwar ein sehr wichtiges Werkzeug, um Nutzer zu sicheren Passwörtern zu zwingen. Durch das Bequemlichkeitsproblem des Nutzers können allerdings immernoch einfach zu erratende Passwörter entstehen. Verhindern lässt sich dies nicht ganz. Die Informationen die Nutzer beim Anmelden bekommen, nutzen Angreifer dann zum Knacken jener Passwörter. Die Faustregel lautet: Je mehr Metadaten, desto besser. (Aus Sicht des Angreifers)

\subsection{Übertragungsproblem}
Das Problem mit der Unsicherheit von Passphrasen oder auch Passwörtern beginnt allerdings jedes Mal aufs Neue, sobald man ein Passwort eintippt. So ist die Bedrohung nicht mit der Wahl eines mathematisch sicheren Passwortes gebannt. Das wissenbasierte Verfahren, welches sich einer Zeichenkette bedient die man in ein Feld eintippt, ist per sé dann unsicher sobald einer der Geheimnisträger (Menschen mit Kenntniss über die Parole) kompromittiert bzw. infiziert ist. So gibt es verschiedenste Angriffsvektoren um das Passwort eines Users für einen speziellen Dienst herrauszufinden. Von personalisierten (oder auch allgemeinen) Phishing Mails, zu Shoulder Surfing bis hin zu Trojanern und Keyloggern auf dem System Desjenigen. Diese können vom Angreifer teilweise remote ausgeführt werden, für manche Angriffe benötigt der Angreifer allerdings physischen Zugriff aufs System. Ein weiteres großes Problem ist die Übertragung von Passwörtern über die klassiche User-Browser-Schnittstelle. Dabei wird das Passwort im Browser des Clients gehasht und dann an den Server übertragen. Die sichere Kommunikation anhand des \ac{https} findet erst bei der Übertragung zum Server statt, die Eingabe des Passwortes an den Browser allerdings ist ungeschützt. Diese Übertragung von Buchstaben kann mitgelesen werden. Das Hashen löst das Problem das ein Angreifer im selben Netzwerk mitlauschen und das Passwort über einen sogenannnten Man-in-the-middle Angriff entwendet. Somit sich der Identität des Passwortbesitzers bedient und sich bei anderen Diensten als Diesen ausgibt. Es bleibt allerdings immernoch das Problem des Wiederholungsangriffs. Der Angreifer muss womöglich also nicht ein Mal das Passwort im Klartext lesen können. Es genügt, den Hash und den Benutzernamen im Request abzufangen um diese dann in einem seperaten Aufruf vom eigenen Rechner an die selbe \ac{url} zu senden. Diese Art des Angriffs nennt man einen Replay Attack. Es handelt sich um das Imittieren von Benutzereingaben durch einen Angreifer, bei der der Angreifer die Passphrasen nicht im Klartext kennt.
Durch den Zugang zum Dienst ist es ihm somit (je nach Implementierung des Dienstes) möglich, sensible Daten des Nutzers einzusehen die nicht für den Angreifer bestimmt sind. Die Frage nach der 'Relevanz' von sensiblen Daten sollte obsolet werden, wenn man an die Möglichkeiten denkt, die der Angreifer mit ihnen nun in der Hand hällt. Mit diesen könnte er den Nutzer zum Beispiel erpressen um an noch mehr Daten oder Geld des Nutzers zu kommen.

\subsection{FIDO2 - Alternative zum Passwort}
FIDO steht für 'Fast Identity Online' (Schnelle Identität im Netz). Sie ist das Ergebnis einer Kooperation des \ac{w3c} und der FIDO Alliance. FIDO2 basiert auf vorhandenen Protokollen wie \ac{webauthn} für die Browser-Server-Kommunikation und CTAP für die Browser-Authenticator-Kommunikation. Auf der offiziellen Webseite der yubico, einem der Hauptentwickler und Publizierer des Vorgängerprotokolls \ac{u2f} wird die FIDO2 - \ac{u2f}, also die zwei Faktor Authentifizierung spezifiziert durch \ac{u2f} das immernoch im FIDO2 Protokoll beheimatet ist, wie folgt beschrieben: ''an open authentication standard that enables internet users to securely access any number of online services with one single security key [...]. FIDO2 is the latest generation of the \ac{u2f} protocol`` [5]. Während das Vorgänger - Protokoll \ac{u2f} von Google und Yubico ins Leben gerufen wurde, ist FIDO2 ein offener dezentraler Kommunikationsstandart für die passwortlose Kommunikation welches die Authentifizierung für sowohl Privatnutzer als auch Unternehmen beqeuem und gleichzeitig sicher machen soll. \\ \\
\includegraphics[width=15cm]{Graphics/FIDO2-Graphic-v2.png} \\
\\
Um die Sicherheit des Nutzers zu gewährleisten kombiniert FIDO2 die Methoden des UAF und des U2F. Bevor wie bei gängigen Zweifaktoren wie eines PINs oder sechsstelligen Schlüssels der Schlüsselaustausch stattfinden kann, muss der User der Anwendung oder des Dienstes eine lokale Verifikation durchführen. Diese soll sicherstellen, dass es sich bei der Person, die die Authentifikation durchführt und der Person, die den Schlüssel vorher registriert hat, um die selbe Person handelt. Diese Verifikation kann zum Beispiel ein Knopf auf einem USB - Stick sein, auf den der Web-Service wartet - Bevor er die Challenge an den Nutzer sendet. So lange also ein Angreifer keinen physischen Zugang zu diesem USB - Stick erhält, ist das Verfahren sicher. Sollte es doch vorkommen, dass der Angreifer Zugriff auf den Stick erhält und den Knopf drückt, setzt die sozusagen zweite Phase ein. Die Webseite sendet dem User eine Challenge, welche der User lokal mit seinem Schlüssel auf dem Computer lösen kann. Die Webseite erhält dann das Ergebnis und vergleicht dieses mit dem eigenen Ergebnis. Gibt es ein Match, sendet der Server der Webseite die zugehörige Response zur Challenge an den User zurück und lässt ihn passieren. Wie die Abbildung zeigt gibt es neben der externen Authentifikation durch Smart-Watch, USB- Stick oder Smartphone auch die Option der 'on-device-authentication'. Damit ist die Authentifizierung durch einen PIN oder einen eingebauten Fingerabdruck-Sensor (über biometrische Daten aller Art) gemeint, die allerdings nicht extern angeschlossen ist sondern sich auf dme Gerät befindet. Auf die initiale Schlüsselerstellung und weitere Details zum FIDO Protokoll, die für diese Arbeit relevant sind, gehe ich im nächsten Kapitel: 'Grundlagen' ein.

\section{Zielsetzung}
Ziel dieser Abschlussarbeit ist es einen Nachweis dafür zu liefern, dass Authentifikation in 2020 sowohl sicher als auch bequem sein kann und das sich diese Punkte nicht gegenseitig ausschließen. Ziel ist es auch, den Leser in die Sicht des Angreifers auf Systeme einzuweisen, sodass im Idealfall automatische Schutzreaktionen wie das Wählen von sicheren Passwörtern hervorgerufen, wenn nicht sogar eine der beschriebenen FIDO2 Verfahren wie der erste oder sogar der Zweite Faktor, verwendet werden. Der Prototyp soll die verschiedenen Authentifizierungsmöglichkeiten veranschaulischen und präsentieren, um dem Nutzer die Wahl auf eines der Verfahren zu erleichtern. Gleichzeitig ist natürlich ein hauptsächliches Ziel dieser Arbeit auch die Grenzen von 'modernen' Authentifizierungsverfahren aufzuzeigen und auch zu zeigen, inwiefern eine Kombination dieser vorhandenen Verfahren, Probleme löst. Ein weiteres Ziel soll es sein, in gewisser weise kategorisch darzulegen, welches Verfahren für welchen Nutzertypen geeignet ist. Die Idee dahinter ist, dass es eine klare Trennung in der Nutzung von Accounts zwischen Entwicklern, Unternehmern und dem 'Casual Websurfer' gibt. Was alle drei Arten von Computernutzern allerdings gemeinsam haben ist: Sie besitzen persönliche Daten, an die kein Angreifer bzw. eben kein Dritter gelangen soll. Der Schutz dieser Daten sollte jedem Individuum selbst wichtig sein, um in den nächsten 5 bis 10 Jahren auf Besserung zu hoffen. Nicht nur technisch muss die Menscheit mit dem neuen digitalen Zeitalter umgehen und sich absichern können, sondern auch auf die menschliche Komponente achtgeben. Es sollte dennoch erwähnt sein, dass diese Arbeit nicht darauf abzielt jede einzelne Authentifikationsstrategie und Möglichkeit aufzuzeigen und zu bewerten sondern eher die klassichen und weiterverbreitetsten Verfahren aufzuzeigen, auf die die restlichen Verfahren meist basieren. So ist der Yubikey am Ende auch nur eine Möglichkeit zur Umsetzung von einer Challenge-Response-basierten Authentikation anhang von privaten Schlüsseln welches bereits im FIDO Standart definiert ist. Auch ist es ein Nicht-Ziel dieser Arbeit das Resultat auf eine einzige perfekte Lösung zu dezimieren und diese zum neuen Standart zu erklären. Viel mehr möchte ich meinen Lösungsansatz präsentieren und disskutieren und darauf aufbauend Vorschläge für die Nutzung von bestimmten Personengruppen abgeben.

% !TeX root = ../my-thesis.tex
\chapter{Grundlagen}
Laut des IT-Grundschutz-Kompendiums vom \ac{bsi} könne ein Benutzer aus Bequemlichkeit oder pragmatischen Gründen bewusst auf komplizierte und unhandliche Kryptomodule verzichten und Informationen stattdessen im Klartext übertragen. \cite{A1} Dies stellt ein hohes Sicherheitsrisiko für Unternehmen, aber auch Privatpersonen dar, da Benutzer nicht gewillt sind ihre Passwörter durch komplizierte Verfahren zu erzeugen und in regelmäßigen Abständen vollkommen randomisiert zu setzen. An neuartige Ansätze des Logins in nativen oder webbasierten Anwendungen stellen sich dadurch völlig neue Herausforderungen. So müssen neue Authentifizierungsmöglichkeiten nicht nur sicher sein, sondern auch komfortabel genutzt und bedient werden können, da sie sonst von den Endnutzern gemiedern oder umgangen werden. Wichtig ist eben auch, dass die breite Masse Zugriff auf die Ressourcen hat, die es zur Nutzung dieser Verfahren braucht. Man denke nur an die ganzen betrieblichen Passwörter, bei denen zum Monatsende nur eine Zahl an der letzten Stelle des Passwortes geändert wird. Laut einer Statistik von 2019, der "Global Data Risk Report From the Varonis Data Lab", gaben 38\% aller Nutzer an ein Passwort im Unternehmen zu nutzen, dass sie nicht (oder nur geringfügig) ändern. Außerdem wird laut dieser Statistik alle 364 Tage wahrscheinlich ein Data-Breach aufgrund von unsicheren Passwörtern in einem mittelständigen Unternehmen stattfinden. Die monatliche Ablaufzeit von Passwörtern in Unternehmen scheint also nicht ganz den Effekt zu erzielen, der ursprünglich damit geplant war, da die Arbeiter die Bequemlichkeit über die Sicherheit stellen.

\section{IT-Grundschutzkriterien}
Der IT-Grundschutz definiert den Schutzbedarf eines bestimmten Assets je nachdem welches Risiko bei Verletzung der Grundwerte Vertraulichkeit, Integrität und Verfügbarkeit entstehen [10]. Allgemein existieren folgende Schutzbedarfskategorien:
\begin{itemize} 
\item \textbf{normal} (Schadenauswirkungen begrenzt bis überschaubar)
\item \textbf{hoch} (Schadenauswirkungen könnten hoch bzw. beträchtlich sein)
\item \textbf{sehr hoch} (Schadensauswirkungen können ein existenziell bedrohliches Ausmaß annehmen
\end{itemize}
Bei dem Authentifikationsprototypen wird die Schadensauswirkung für alle nicht personenbezogenen Daten wohl im Bereich normal liegen, da schon im Aufbau darauf geachtet wird, dass nur so viele Daten vom User verwendet werden, wie zwingend notwendig. (Nach dem Need-To-Know Prinzip) Sollte die Webseite oder Teile der Webseite publiziert bzw. im Business - Umfeld genutzt werden, muss eine Neubewertung der Daten nach Vertraulichkeit, Integrität und Verfügbarkeit stattfinden. Vor allem muss darauf geachtet sein, dass Daten über Fingerabdrücke verschlüsselt und Passwörter im gehashten Zustand in Datenbanken persistiert werden. Bei Kompromittierung des Hauptrechners, welches die größte Bedrohung in diesem Szenario darstellen würde, droht ein Data Breach mit dem Angreifer diese Daten weiterverwenden können. Durch Verschlüsselungen durch Schlüssel, die nicht auf dem Hauptsystem (oben u.a als Hauptrechner benannt) liegen. Gleichzeitig sollten die genutzten Verfahren insgesamt mathematisch sicher sein, auf veraltete Verschlüsselungsverfahren ist zu verzichten. Dies gillt auch für Hashes wie MD5, die mittlerweile relativ akkurat durch sehr große vorgerechnete Tabellen erraten werden können.

Im Falle einer Kompromittierung besäße der Schutzbedarf der Daten innerhalb der Datenbank, welche nicht gehasht oder anderweitig verschlüsselt sind, die Kategorie 'sehr hoch'. Eine Kompromittierung kann für das Unternehmen einen Imageschaden sowie weitreichende juristische Klagen zur Folge haben. Damit sind bereits 2 der 7 aufgeführten Schadensszenarien durch den \ac{bsi} beschrieben. Je nach dem wie kompliziert die Ursprungsdaten sind und welcher Hashingalgorithmus in Kombination mit Salt und Pepper genutzt wurde, können vorallem Nutzerpassphrasen an die Öffentlichkeit gelangen und Angreifer können die Passwörter für andere Dienste nutzen. Die Wahrscheinlichkeit für diesen Schaden ist noch vergleichbar niedrig, weshalb die Kategorie nach dem \ac{bsi} für Passwörter in dem Prototyp hoch statt sehr hoch ist.
\newpage

\includegraphics[width=15cm]{Abb_2_09_Varianten.png}

Außerdem unterscheidet der IT-Grundschutz drei Arten der Absicherung. Die Basis-Absicherung ist relevant für Institutionen die einen Einstieg in den IT-Grundschutz suchen und relativ schnell alle relevanten Geschäftspozesse mit einfach umzusetzenden Basismaßnahmen sichern wollen. Die Kern-Absicherung konzentriert sich auf besonders wichtige Geschäftsprozesse und vertieft sich in die Sicherung dieser. Von einer Standard-Absicherung spricht man, wenn alle empfohlenen IT-Grundschutz-Vorgehensweisen durchgeführt werden. Sie beschreibt den allumfassenden Schutz der Prozesse und Bereiche der Institution, wie das Schaubild vom BSI verdeutlicht. [9] Bei dem Prototyp wird eine Basis-Absicherung nach BSI durchgeführt und darauf geachtet alle Datenschutzkriterien zu erfüllen. Dabei wird vor allem die Checkliste des IT-Grundschutzes zu Webservern und Webanwendungen betrachtet. Kriterien die nicht erfüllt werden konnten oder wurden, werden dokumentiert und im Fazit erläutert. Die Wahl der Basis-Absicherung begründet sich damit, das lokale Testdaten im Prototyp verarbeitet werden, für die kein bis nur ein sehr geringer Schutzbedarf besteht. Eine Kern-Absicherung käme nur in Frage, falls ein ganz bestimmter Prozess oder Asset des Prototyps geschützt werden müsste wie zum Beispiel der Zugriff auf die Datenbank durch den Prototypen, welche nicht der Fall ist. Insgesamt ist dieser Teil der Abschlussarbeit auch als Einstieg in den IT-Grundschutz zu verstehen, welcher laut BSI selbst die Basis-Absicherung als Empfehlung und zur Folge hat und sich am Besten für vorhandene Zwecke eignet.

\section{Standart nach FIDO2}
Die genannten Probleme sind den Menschen seit einigen Jahren bekannt und wurden vor allem mit Verfahren gelöst, die keine Passwörter (oder allgemein Verfahren der Wissenskategorie) zur Authentifikation benötigen und diese maximal als ersten Layer der Sicherheit implementieren. Das bekannteste Beispiel für einen Standart zur sicheren und bequemen Authentifikation im Internet findet man unter dem Schlüsselwort FIDO2. FIDO steht dabei für 'Fast Identity Online' (Schnelle Identität im Netz). Sie ist das Ergebnis einer Kooperation des \ac{w3c} und der FIDO Alliance. FIDO2 basiert auf vorhandenen Protokollen wie \ac{webauthn} für die Browser-Server-Kommunikation und CTAP für die Browser-Authenticator-Kommunikation. Auf der offiziellen Webseite der yubico, einem der Hauptentwickler und Publizierer des Vorgängerprotokolls \ac{u2f} wird die FIDO2 - \ac{u2f}, also die zwei Faktor Authentifizierung spezifiziert durch \ac{u2f} das immernoch im FIDO2 Protokoll beheimatet ist, wie folgt beschrieben: ''an open authentication standard that enables internet users to securely access any number of online services with one single security key [...]. FIDO2 is the latest generation of the \ac{u2f} protocol`` [5]. Während das Vorgänger - Protokoll \ac{u2f} von Google und Yubico ins Leben gerufen wurde, ist FIDO2 ein offener dezentraler Kommunikationsstandart für die passwortlose Kommunikation welches die Authentifizierung für sowohl Privatnutzer als auch Unternehmen beqeuem und gleichzeitig sicher machen soll. \\ \\
\includegraphics[width=15cm]{Graphics/FIDO2-Graphic-v2.png} \\
\\
Um die Sicherheit des Nutzers zu gewährleisten kombiniert FIDO2 die Methoden des UAF und des U2F. Bevor wie bei gängigen Zweifaktoren wie eines PINs oder sechsstelligen Schlüssels der Schlüsselaustausch stattfinden kann, muss der User der Anwendung oder des Dienstes eine lokale Verifikation durchführen. Diese soll sicherstellen, dass es sich bei der Person, die die Authentifikation durchführt und der Person, die den Schlüssel vorher registriert hat, um die selbe Person handelt. Diese Verifikation kann zum Beispiel ein Knopf auf einem USB - Stick sein, auf den der Web-Service wartet oder ein Fingerabdruck-Sensor - Bevor er die Challenge an den Nutzer (bzw. über dessen Browser dann an die Webseite) sendet. So lange also ein Angreifer keinen physischen Zugang zu diesem Gerät erhält, ist das Verfahren sicher. Sollte es doch vorkommen, dass der Angreifer sich Zugriff verschafft und den Knopf drückt, setzt die sozusagen zweite Phase der Authentifikation ein. Die Webseite sendet dem User eine Challenge, welche der User lokal mit seinem Schlüssel auf dem Computer lösen kann. So authentisiert sich der User gegenüber dem Webserver, auf dem die Webseite gehostet ist. Die Webseite vergleicht das Userergebnis dieses mit dem eigenen Ergebnis. Gibt es ein Match, sendet der Server der Webseite die zugehörige Response zur Challenge an den User zurück. Die Webseite hat den User authentifiziert. Wie die Abbildung zeigt gibt es neben der externen Authentifikation durch Smart-Watch, USB- Stick oder Smartphone auch die Option der ''on-device-authentication``. Damit ist die Authentifizierung durch einen PIN oder einen eingebauten Fingerabdruck-Sensor (über biometrische Daten aller Art) gemeint, die allerdings nicht extern angeschlossen ist sondern sich auf dem Gerät selbst befindet.

\section{Behandelte Verfahren}
\subsection{Username \& Passwort}
Trotz der bereits erwähnten Probleme des Passwortes, ist es laut eines wissenschaftlichen Artikels von Thomas Maus 2008 ``nicht aus unserer Arbeitswelt [...] wegzudenken'' [12]. Anzumerken ist, dass der Artikel bereits 12 Jahre in der Vergangenheit liegt und immer noch Relevanz hat. Herr Maus beschreibt alternative Authentifikationsmethoden wie die Einmalkennwörter und biologische Merkmale. Dies ist nur ein weiterer Beweis dafür, dass schon vor mehr als 10 Jahren die Passwortproblematik erkannt wurde. In dem Artikel geht Herr Maus der Hypothese nach, ob Passwörter wirklich per sé unsicherer sind als die Authentifikationsmethoden der beiden anderen Kategorien Besitz und biologische Merkmale. So sei das Kernproblem des Passwortes, dass es direkt nach der Eingabe ein geteiltes Geheimnis ist, da alle beteiligten Systeme es mitschneiden konnten und automatisch Geheimnisträger sind. Das Passwort bietet sehr viele Angriffsvektoren, so ``Shoulder Sourfing, Phishing, Social-Engineering, Man-in-the-Middle Angriffe, schlechte Passwort Qualitäten, das Teilen des Passwortes mit Familie und Freunden'' [12] und vielem mehr.

Bei dem Prototyp wird der Ist-Zustand einer Username und Passwort - Authentifikation demonstriert, wie man sie heutzutage auf höchstwahrscheinlich jedem Webdienst in der Form als ersten schwachen Faktor vorfinden wird. Teilweise wird sie sogar als einziger Authentifizierungsfaktor verwendet. Während man über alle Schwächen des Passwortes spricht, muss man dennoch anerkennen, dass das Passwort eine gewisse Flexibilität bietet. Es genügt das Wissen über eine bestimmte Zeichenfolge um sich zu authentifizieren, dieses Wissen muss nicht zwangsmäßig auf ein Blatt geschrieben oder in einem Passwort-Manager beherrbergt werden. Das beste Passwort ist jenes, welches nur in dem Gehirn des Nutzers persistiert ist und mit niemandem geteilt wird. Anders als bei neueren Verfahren, die als sicherer gelten, benötigt es keine Smart-Card, USB-Sticks oder anderweitige technische Hilfsmittel um sich zu authentifizieren. Lediglich das Gedächtnis des Nutzers ist gefragt, welches allerdings fortlaufend im digitalen Zeitalter von Social Media und Up-To-Date Informationen, nachlässt. So ist es Usern heutzutage immer schwerer möglich sich die verschiedenen Passphrasen zu merken, ohne sie als Behelf irgendwo zu notieren. Sobal das Passwort auf eine potenziell unsichere Plattform notiert wurde, gillt sie als geteiltes Geheimnis zwischen jedem Leser und dem Initialbenutzer dessen.

\subsection{OTP's}
Ein Kennwort ist eine "eine Zeichenfolge, die zur Authentifizierung verwendet wird. Damit soll die Identität einer Person [...] auf eine Ressource nachgewiesen werden." \cite{A4}. Ein Einmalkennwort im Vergleich ist ein Kennwort, das nur ein einziges Mal für eine Authentifizierung genutzt werden kann. Man unterscheidet drei Arten von \ac{otp}'s.

\includegraphics[width=15cm]{TOTP-algorithm-explained.png}

Bei der timerbasierten (\ac{totp}) Methode wird die aktuelle Systemzeit (Token time) mit dem zu verschlüsselnden Text (Shared secret) anhand eines kryptografischen Verfahrens verschlüsselt. Es entsteht ein kryptografischer Hashwert (Cryptographic Hash), welcher meist den HMAC-SHA1 Hashingalgorithmus verwendet. Dieses Verfahren findet gleichermaßen auf dem Server statt. Der einzige Unterschied zum Server besteht darin, dass die Systemzeit des Servers genutzt wird. Findet innerhalb der festgelegten Zeit (ein Token entfällt laut RFC6238 standartmäßig nach 30 Sekunden, diese Zeit ist modifizierbar) eine erfolgreicher Match auf Serverseite statt, wird der User authentifiziert und ihm der Zugang gewährt. [13] Ein entscheidender Nachteil dieser Methode entsteht, falls der authentifizierende User innerhalb dieser 30 Sekunden (beispielsweise) durch eine Störung des Netzes oder einen kompletten Netzausfall die Verbindung verliert. Der aktuelle TOTP Code wird ungülltig und muss erneut angefragt bzw. erstellt werden. Apps wie der Google Authenticator lösen diesen Problem, in dem sie einen Timer setzen und alle 30 Sekunden einen automatisch generierten neuen Code mit der aktuellen ablaufenden Zeit des Timers anzeigen. Dabei muss natürlich drauf geachtet werden, dass Server und Client synchron sind, dieses Verfahren ist deshalb an eine bestimmte Zeitspanne und ein externes Gerät (sei es Smartphone, TOTP Smart Card oder ein externer Rechner) gebunden. Diese Zeitbegrenzung bzw. das Verlassen auf die Zeit des Gerätes und des Servers kann zum Nachteil werden wenn Geräte asynchron werden. So gibt es einige ältere TOTP - Generatoren, die einen internen Zähler haben und mit jedem Tick eine Sekunde auf den aktuellen Wert nach UTC draufrechnen. Find bei Geräterestart keine erneute Synchronisation (im RFC6238, Resynchronisation) statt, so entstehen fehlerhafte TOTP - Codes auf Clientseite, die der Server ablehnen wird. Dieser Nachteil ist gleichzeitig allerdings ein großer Vorteil in punkto Sicherheit, da ein Angreifer potenziell einen sehr kleinen zeitlich begrenzten Angriffsvektor besitzt, in dem er den TOTP - Code lösen muss. Durch entsprechende Zugriffssicherheitsfeatures, wie das Sperren des Nutzers nach oftmalig wiederholter falscher Eingabe, wird ein Bruteforce-Angriff verhindert.

Ereignisbasierte OTPs (folgend \ac{hotp}) besitzen einen Ereigniszähler, der bei jeder versuchten Authenzifizierung einen Zähler auf Server und Clientseite synchronisiert inkrementiert. Sollte der Zähler asychnron werden bzw. der Server einen anderen Wert gespeichert haben als der Client bei der nächsten Authentifizierung sendet, wird der Authentifizierungsvorgang abgebrochen. Man findet diese Funktionalität wortwörtlich beschrieben in Googles Time and event based one time password" - Patent \cite{A6} in folgendem Wortlaut "[...] the characteristics of an event can be the value of a counter that is incremented each time the user pushes a button on the token" \cite{A6}. Für diesen Prozess wird ein spezieller Algorithmus genutzt, der im RFC4226 näher beschrieben ist.

Challenge-response basierte OTP Verfahren bedienen sich an komplizierten mathematischen Verfahren. Das heißt, es erfolgt ein ACK (Acknowledge bzw. Initialanstoß zur Authentifizierung). Der Client, berechnet die Response mithilfe der mathematischen Formel und sendet das Ergebnis an den Server. Sollte es einen Match geben, erhält der Client eine Response vom Server, der seine Echtheit bestätigt. Synchronisationsprobleme kann es bei diesem Verfahren entgegen der ereignis oder timerbasierten OTP-Verfahren nicht geben, da die Berechnung dieses 'Schlüssels' vollkommen auf der Clientseite funktioniert. Der Server überprüft diese Rechnung nur mit seinem eigenen Wert, stellt aber keine weiteren Rechnungen oder Umformungen mit diesem Wert an. Der Hauptvorteil dieses Verfahrens ist, dass unabhängig von der Zeit und einem speziellen Ereignis eine Anfrage gestellt werden kann. Der Server kann also seine 'Challenge' abschicken und muss keine 'Response' innerhalb einer festgegebenen Zeit erhalten, um authentifizieren zu können. Dieses Verfahren gilt als besonders sicher, da es auf Serverseite keinen Algorithmus gibt, der sich vorausberechnen lässt.

Die folgende Tabelle soll vergleichend die Gemeinsamkeiten, aber auch Vor und Nachteile von HOTP mit TOTP verdeutlichen. Das Wissen zu vor allem den Nachteilen der Verfahren ist wichtig um die gewählte Eigenstrategie im Prototypen zu verstehen.

\includegraphics[width=15cm]{Tmp.png}

\subsection{WebAuthn}
Webauthn ist kurz für die 'Web Authentication'. Zur Verfügung gestellt wurde dieser Standard der Authentifikation 2018 von der FIDO Alliance und dem W3C. \cite{A7} Sie ermöglicht eine passwortlose (bzw. benutzerdatenlose, es wird also auch keine User-ID benötigt) Authentifikation durch Tokens (Sicherheitsschlüssel). Für dieses Verfahren wird zunächst ein Buffer aus kryptografischen random Bytes generiert, dass der Verhinderung von 'Bruteforce' (vom webauthn - Guide auch als 'reply attacks' beschrieben) - Angriffen dienen soll. Web Authentication nutzt das vorhandene public-key-Verfahren für Webseiten. Der Standard definiert allerdings nicht welche Art von Schlüssel genutzt wird. Unter den Möglichkeiten zählen der "USB security key" \cite{A7} oder der "built-in fingerprint sensor" \cite{A7}. Webauthn basiert auf vielen bereits vorhandenen Abhängigkeiten der Informatik wie die Standards von HTML5, ECMAScript, COSE (CBOR Object Signing and Encryption COSE, RFC8152) oder dem Nutzen von der Base64url encoding.

Zusammengefasst hat der FIDO2 Standard mit CTAP (dem Protokoll für externe Authentifikationen mit Mobilgeräten) und Webauthn (der Schnittstelle bzw. "API") vorhandene Funktionen definiert, mit der native Authentifizierungsmethoden wie das public-private Key Verfahren auf die Webseite übertragen werden können. Wichtige Beispiele für Web Authentication sind der Yubikey, der USB-Token oder unsere biometrischen Daten (FaceID oder TouchID), die wir täglich in jedem AppStore nutzen, der diese Daten verschlüsselt an eine Webseite bzw. einen öffentlichen Store übermittelt. In unserem Use-Case wollen wir die Authentifizierung anhand von Webauthn als Multi-Faktor nutzen, also als zusätzliche Sicherung neben einem Passwort. Denn wie oben beschrieben, kann das Wissen eines Menschen durch Data Breaches oder menschliches Versagen (Gutglauben) leicht abhanden kommen. Mit einem Yubikey oder einem USB-Token befindet man sich allerdings auf der Ebene des Besitzes, wodurch ein potenzieller Angreifer es schwerer hat an die Daten zu kommen. Biometrische Daten und diese public-private-Key Verfahren gelten nämlich allgemein als sehr sicher. Wodurch es technisch sehr schwer bis mathematisch (in gegebener Menschenzeit) fast unmöglich ist, die Algorithmen hinter ihnen zu knacken.

% !TeX root = ../my-thesis.tex
\chapter{Methodik}
\section{Aufbau des Prototypen}
Das Ziel des Prototypes ist es, wie eingangs erwähnt, vorhandene Authentifizierungsverfahren abseits der klassischen UserID / Passwort Methode zu begutachten und dessen Schwächen aufzudecken. Darauf aufbauend wurde die Methodik für die Findung eines Verfahrens welches die Kriterien der Sicherheit, des Datenschutzes und der Bequemlichkeit besser erfüllt, gewählt. Der Prototyp beschreibt eine einfache Webseite, die aus dem Internet erreichbar sein wird und zwei Eingabefelder und einen Loginbutton besitzt. Die unterschiedlichen Methoden der Authentifizierung wählt man über ein Dropdownmenü, welches sich zwischen Login-Button und Username / Passwort Feld befindet. Je nach Authentifizierungsverfahren werden kleine Popup-Boxen sichtbar, die die weiteren Schritte für die Authentifikation erläutern.
Ein Teil des Prototypen soll die gewählten Methoden demonstrieren. Neben einer erfolgreichen Demonstration der Authentifizierung sollen nach jeder Methode auch Kennwerte ausgegeben werden. Einer davon soll zum Beispiel die Zeit von der ersten Eingabe in ein Eingabefeld bis zur Authentifizierung zählen und anzeigen. \\
\\
Neben den vorhandenen Methoden soll die eigene Architektur aufgebaut werden, die sich an vorhandenen Authentifikationsmöglichkeiten bedient. Die UserID und das Passwortfeld bleiben stets im Vordergrund, müssen allerdings nicht zwingend für jedes der Verfahren genutzt werden, so kann es zum Beispiel bei einer besitzbasierten Authentifikation bereits reichen, den Besitz (z.B einen USB, welcher einen Schlüssel beherbergt) im Computer einstecken zu haben und auf den Loginbutton zu drücken.

\section{Auswahl der Authentifizierungsverfahren}
Neben des altbekannten \ac{totp} Verfahrens, wird das Secure Element und die E-Mail Authentifikation betrachtet. Dabei bedienen sich diese Verfahren aller drei Möglichkeiten der Authentifikation, dem Wissen, dem Besitz und der körperlichen Merkmale. Das Dropdownmenü zeigt die Authentifikation über biologischen Merkmale (Touch ID oder Face ID) nur sofern das Gerät, auf welchem der Prototyp bedient wir, einen entsprechenden Sensor und die entsprechende Software zur Verarbeitung besitzt.

\section{Kriterien zur Bewertung des Prototypen}
\section{Architektur}

% !TeX root = ../Bachelorarbeit.tex
% \chapter{Prototypischer Lösungsansatz}

\section{Implementationsdetails}
\begin{enumerate} 
\item \textbf{Client / Webseite}

Die Webseite besteht aus einer zur \ac{spa} ähnlichen Architektur, die ausschließlich Javascript und JQuery, also clientseitige Programmiersprachen nutzt. SPA's sind Webapplikationen, bei denen der Nutzer eine Seite betritt und diese nie wieder vollständig laden muss. Frameworks wie Angular, React oder VueJS basieren auf dieser Mechanik und nutzen Javascript und JQuery um die Ladezeiten einer Webseite durch Module zu reduzieren. Anstatt also die gesamte Webseite neuzuladen, laden diese Frameworks nur spezielle Bereiche der Webseite asynchron nach. In meinem Prototyp ist dies größtenteils der Fall, da neben der Hauptseite \textbf{index.html} eine weitere Seite existiert, auf die man bei erfolgreichem Login weitergeleitet wird: \textbf{secret\_panel.html}. Auch werden keine Module nachgeladen, sondern bei entsprechender Aktion nur Elemente innerhalb des DOM - Baums per id identifiziert und versteckt.

Beim Betreten von beiden Seiten der Anwendung findet eine Prüfung nach dem Session-Cookie (\textbf{user\_sid}) statt, die vom Server bei einer erfolgreichen Authentifikation im Header über 'Set-Cookie' als Response zurückkommt und vom Browser in folge dessen gesetzt wird. Der Aufruf der \textbf{index.html} Seite ist nicht möglich mit gesetztem Cookie und leitet den User auf \textbf{secret\_panel.html} weiter und umgekehrt genauso. Die Sicherung dieses Cookies auf Serverseite durch Prüfungen ist nicht Bestandteil dieser Arbeit, hier wird sich ledeglich auf den Loginprozess von Anwendungen konzentriert um dessen Schwächen und die Vorteile neuer Verfahren aufzuzeigen. Aktuell könnte ein Angreifer demnach selbst einen entsprechenden Cookie mit einem Wert setzen und das 'geheime Panel' erreichen. Bei weiter ausgereiften Webapplikationen wird bei jedem Request an den Server der Session Cookie mit einer temporären Map innerhalb des Backends abgeglichen. Befindet sich der Session Cookie nicht in dieser Map oder hat nicht genügend Rechte für diese Anfrage, erhält der Nutzer den Statuscode 401 (Unauthorized) zurück. So haben es etablierte Dienste wie 'Netflix' und 'Microsoft' bereits umgesetzt.

Aus der Forschungsfrage ergibt sich, dass die Webseite ausschießlich Javascript (oder auf Javascript basierende Programmiersprachen wie JQuery) zur Implementation der Programmlogik verwendet. Javascript selbst ist eine clientseitige Programmiersprache und gibt dem Nutzer den gesamten Code auf Webseiten preis. Demnach wurde bei der Anwendung darauf geachtet, alle clientseitigen Prüfungen, auch serverseitig zu implementieren. Zumindestens alle wichtigen Prüfungen, die sonst den Programmablauf verhindern würden oder sogar Serverabstürze zur Folge hätten. Nicht nur gibt Javascript den Code preis, über die Konsole (Unter Google Chrome und Firefox in den Entwickleroptionen des Browsers zu finden) lassen sich ganze Funktionen oder globale Variablen manipulieren. Variablen innerhalb von Funktionen können vom Angreifer nicht so leicht manipuliert werden.

Möglich ist es dennoch, den Inhalt dieser Funktion in eine andere Funktion zu schreiben, um die Variable dort auszutauschen. Um dem Nutzer nicht jede Funktion problemlos erreichbar zu machen, wurde bei der Implementation das Revealing-Module-Pattern angewendet, über die gewisse Variablen nicht im globalen Scope (window) sondern im Scope einer Funktion bleiben, die nur gewisse andere Funktionen in sich exportiert und als eine Art Schnittstelle fungiert. Auf dieses Pattern wird im Folgenden näher eingeggangen.

Das Design der Webanwendung ist kein wesentlicher Punkt dieser Arbeit und deshalb schlicht gehalten. So besteht der Prototyp aus einer einfachen Webseite, die aus dem globalen Internet erreichbar sein wird und die zwei Eingabefelder und einen Loginbutton besitzt. Um sich die Designarbeit oder das Suchen von Icons und passenden Buttondesigns zu sparen wurde auf bekannte Frameworks wie Bootstrap4 und Funktionalitäten wie Flexbox gesetzt, die eine automatisches Rescaling und Positioning von Webseitenelementen beinhalten. So war es nicht nötig ein Loginformular zu designen sondern vorhandene Strukturen von Bootstrap für Buttons aller Art zu nutzen. Die unterschiedlichen Methoden der Authentifizierung wählt man über ein Dropdownmenü über dem Login - Button. Je nach Authentifizierungsverfahren werden kleine Popup-Boxen sichtbar, die die weiteren Schritte für die Authentifikation erläutern. Die Methode kann sowohl über das Dropdownmenü als auch über einen Klick auf die Pfeile links und rechts neben dem Dropdownmenü gewählt werden, die als Pagination zwischen den verschiedenen behandelten Verfahren fungiert. Ist man am Ende der drei Methoden, springt man wieder zur ersten Methode und andersherum.

Bei der Web Authentication gibt es eine Besonderheit. Die Webseite auf die Schnittstellen des Betriebssystems zugreifen, um den Nutzer zu verifizieren. Dieser Teil kann von meinem Prototyp nicht beeinflusst werden und wird vom CTAP2 innerhalb des FIDO2 Standards definiert. Dadurch entstehen teilweise merkwürdige und aus UX (User Experience) - Sicht höchst fragwürdige Interaktionen. So fragt das Betriebssystem zunächst (bei entsprechender Möglichkeit) nach einem PIN um den Nutzer zu registrieren. Drückt man nun die Escape-Taste erscheint ein Dialog um einen Sicherheitsschlüssel (ein externes Gerät) einzurichten. Beim Login wiederrum ist dies durch eine Dropdownliste schöner gelöst worden, wo der User alle möglichen Loginmethoden auf einem Blick sieht und diese Wählen kann. Während er bei der Registration keine Chance hat dies zu tun und immer erst ein PIN - Feld angezeigt wird. Auf die einzelnen behandelten Verfahren wird in späteren Kapiteln noch genauer eingeggangen, da werden solche Schwierigkeiten aufgegriffen da dies nur eines von vielen 'Problemen' neuerer Verfahren ist: Die Abhängigkeit vom Betriebssystem.

\item \textbf{Server}

Der NodeJS - Server besteht aus einer REST Api, die keine Zustände speichert. Neben der Aufgabe des Cookie Managements und der Generierung von UUID's liefert er zudem die Schnittstelle zur Datenbank und verschiedene Funktionalitäten für die behandelten Verfahren Username \& Passwort, TOTP und Webauthn:

\begin{itemize}
 \item \textbf{/logout} - POST - Löscht den Session-Cookie (und damit die Session des Nutzers) und leitet ihn auf die Loginseite \textbf{index.html} weiter.

 \item \textbf{/get\_public\_key} - GET - Liest den öffentlichen Schlüssel des Servers ein und gibt diesen in einer JSON - Struktur zurück. Ist wichtig für die Username \& Passwort - Authentifizierung.

 \item \textbf{/password/login} - POST - Nimmt einen mit dem öffentlichen Schlüssel des Servers verschlüsselten Usernamen und Passwort entgegen und entschlüsselt diese mit dem privaten Schlüssel. Im Anschluss darauf wird in der Datenbank über einen gepoolte Query (eine Anzahl an Verbindungsprozessen zur Datenbank) nach dem Nutzer gesucht. Wurde dieser gefunden, erstellt der Server einen SHA512 - Hash aus dem Passwort (aus dem Request) und dem Salt des Users (aus der Datenbank) indem die \textbf{hashString} - Methode aufgerufen wird. Diese arbeitet mit der crypto Library um den Hash zu generieren und bezieht den Pepper des Servers mit ein. Sofern die Hashes aus der Datenbank und der soeben Generierte übereinstimmen, erhält der Nutzer die Response 200 und den Text ``OK''. Gleichzeitig wird wie bei jeder anderen Authentifizierungsmethode die \textbf{createSessionCookie} - Methode aufgerufen um den Session-Cookie (eine zufällige UUID) im Header zurück an den Nutzer zu senden, sodass dieser ihn im Browser.
 \newpage
 
 \item \textbf{/password/create\_password\_hash} - GET - Nimmt eine Zeichenkette des Nutzers entgegen und erstellt einen Password-Hash für den Nutzer, der manuell in die Datenbank persistiert werden kann. Ist als 'Quality of Service' Funktion zu verstehen, um es dem Administrator einfacher zu machen, Passwörter zu erstellen, die bei Vergleich einen gülltigen Hash ergeben.
 
 \item \textbf{/totp/check\_username} - POST - Diese Methode dient lediglich der Prüfung, ob der Nutzer die TOTP Authentifikation mittels zufälligem SECRET bereits gegenüber der Webseite validiert hat. Das Datenbankfeld 'totp\_activated' wird hier geprüft, ist der Wert 1 (stehen für das boolsche 'wahr') liefert der Server den Text ``Success'' zurück, das der Client deutet um nur das Feld für den sechsstelligen OTP Code anzuzeigen. Hat das Feld allerdings den Wert 0, wird entweder (sofern noch nicht vorhanden im Feld 'totp\_secret') ein neues Secret erstellt und in der Datenbank für diesen Nutzer persistiert. Gleichzeitig für eine URL beginnend mit otp:// erstellt, um diesen zusammen mit einem Base 64 enkodierten QR Code an den Client geschickt, der dies dem User präsentiert. Es wurde sich hier bewusst dazu entschieden, diese Schritte auf Serverseite vorzunehmen. Möglich wäre es auch auf Clientseite gewesen, wäre aber mit einem Mehraufwand verbunden, welches hier verhindert werden sollte.
 
 \item \textbf{/totp/check\_token} - POST - Nimmt den Usernamen des Nutzers und einen sechsstelligen TOTP Token entgegen. Sucht im Anschluss darauf in der Datenbank nach dem Nutzer und prüft über die Library 'otplib' ob der eingegebene TOTP - Token zum secret in der Datenbank valide ist. Dieser Schritt findet sowohl bei der Registration als auch beim Login statt. Wenn das Feld 'totp\_activated' also 0 ist, wird es bei der ersten Authentifikation auf 1 gesetzt und der Nutzer wird eingeloggt. (Session-Cookie wird gesetzt und nutzer weitergeleitet)
 \newpage

 \item \textbf{/webauthn/generate-attestation-options} - POST
 Liefert dem Nutzer die verschiedenen Registrieroptionen für die Web Authentication mittels WebauthnAPI. Dabei wurden verschiedene Optionen gesetzt, die im folgenden erklärt werden und die der User im Body der Response als JSON - Objekt erhält.
 
 \begin{lstlisting}[language=json,firstnumber=1]
        {
            rpName: PROJECT_NAME,
            rpId: WEBAPP_ORIGIN,
            userID: queryRes.rows[0].id,
            userName: username,
            userDisplayName: username,
            attestationType: 'none',
            authenticatorSelection: {
                requireResidentKey: false,
                userVerification: "discouraged",
            },
            excludedCredentialIDs: userAuthenticators.map(dev => dev.credentialID),
        }
\end{lstlisting}

\textbf{rpName:} rp steht hier bei für Relying Party. Das ist der Name, der später auch im Registrierungsdialog als 'Webseiteninhaber' gelistet wird.

\textbf{rpId:} Die Relying Party Id bekomtm eine URL, auf der Sich der Nutzer registrieren möchte. In unserem Falle ist dies 'localhost' ohne Zusätze wie das Protokoll oder der Port. Diese Variable ist vor allem für Webseiten im Netz wichtig, sodass ein Angreifer nicht per Pishing den Nutzer zur Registrierung auf einer anderen Webseite bringt. Die Registrierung schlägt clientseitig fehl, wenn die rpId und die Webseitenaddresse (Damit ist nicht die Domain sondern die wahrliche IP-Addresse gemeint) nicht übereinstimmen. Beim Wert 'localhost' sendet der Server diesen Teil der JSON - Abfrage nicht mit, da es localhost aus Testgründen nicht prüft. Gleichzeitig erzwingt Webauthn ausschließelich für diese rpId keine sichere Verbindung.

\textbf{userID:} Das ist die ID des Nutzers die verifiziert wird, meist eine fortlaufende Ganzzahl.

\textbf{userDisplayName:} Das ist der Name, der beim Registrieren als Username gelistet wird.

\textbf{attestationType:} Der Server definiert, wie viele Informationen über den Authenticator er im attestation statement haben möchte. Das attestation statement erhält er nach dem der User sich mit einem Authenticator registriert hat und ist ein wichtiger Bestandteil des Objekts welches den Server im Nachhinein über eine erfolgreiche Registration benachrichtigt. Ferner ist der öffentliche Schlüssel des Nutzers in diesem Objekt lokalisiert, den der Server dann in der Datenbank persistiert. 'none' bedeutet hierbei, dass keinerlei Informationen erwünscht sind. 'indirect' als Option würde dem Nutzer erlauben selbst zu entscheiden wie viele Informationen er preisgibt bzw. könnte er eine anonymisierte CA verwenden um sein Zertifikat auszustellen und 'direct' als Option würde die Daten direkt vom Authenticator ohne einen Eingriff des Nutzers erwarten.

\textbf{authenticatorSelection:} Das sind verschiedene Optionen um den verwendeten Authenticator zu bestimmen. Das Argument 'requireResidentKey' bestimmt hierbei darüber, ob ein Authenticator benutzt werden darf, der selbst nicht in der Lage wäre einen privaten Schlüssel auf dem Betriebssystem zu sichern und dem Server in folge dessen einen öffentlichen Schlüssel zu senden. Die 'userVerification' ist im Prototyp nicht erzwungen also 'required' sondern 'discouraged', es obliegt also dem Betriebssystem und dem zu authentifizierenden Gerät / und der initialen Konfiguration zu entscheiden, ob der Nutzer sich gegenüber seines Authenticators verifizieren muss. In dem Beispiel eines bereits sicheren FIDO2 - USB Sticks könnte dies zum Beispiel nicht vom Nutzer gewünscht sein.

\textbf{excludedCredentialIDs:} Dies ist eine Liste von credentialID's die in der Datenbank im Feld 'webauthn\_authenticator\_data' als JSON - Sturktur persistiert sind. Sie dient dazu, bereits registrierte Methoden nicht erneut anzuzeigen, funktioniert dennoch nicht zuverlässig.
\end{itemize}

\item \textbf{Datenbank} (Schadensauswirkungen können ein existenziell bedrohliches Ausmaß annehmen
\end{enumerate}

\section{Erfolgreiche Authentifikation}




% !TeX root = ../Bachelorarbeit.tex
\chapter{Auswertung}
\section{Improved Authentication}
\section{Fehlerbetrachtung}

% !TeX root = ../Bachelorarbeit.tex
\chapter{Ausblick und Fazit}
Die Userverifikation ist eine der größten Herausforderungen an Webanwendungen der Neuzeit und wird es voraussichtlich auch bleiben. Das Verfahren der Negativauthentifikation zeigt in welche Richtung zukünftige Verfahren wohl gehen: Man versucht die alteingesessenen Passwörter durch neue Denkansätze und Verfahren besser zu schützen. In diesem konkreten Beispiel macht man wortwörtlich das Gegenteil einer gewöhnlichen Authentifikation und versucht dadurch falsche Anfragen zu erkennen bevor der Server sie überhaupt prüfen muss. Zusätzlich scheint der Zufall in den Fokus zu kehren, während die gewöhnliche Multi-Fator-Authentifikation feste Auswahlmöglichkeiten bot, ist die A-MFA darauf spezialisiert dem User seinen Parametern zugeschnittene Möglichkeiten zu bieten. Das macht es Angreifern ungemein schwer, einen bestimmten Nutzer konkret anzugreifen - Da jeder Nutzer ein potenziell anderes Preset bzw. 'Profil' besitzen wird, das seine Authentifizierungsmöglichkeiten ergibt. Neben neuen Ansätzen muss sich die zukünftige Forschung wohl vermehrt mit der Haltung der Nutzer zum Thema Bequemlichkeit widmen. Neuere Verfahren, die aus technischer Sicht sehr sicher sind, können nicht bestehen wenn sie vom Nutzer nicht als angenehm bzw. bequem empfunden werden. Die Digitalisierung zwingt Webseitenbetreiber (durch öffentlichen Druck und Konkurrenz) dazu, alle vorhandenen Möglichkeiten zur Authentifikation anzubieten und dem Nutzer die Wahl dessen zu lassen. Das Selbe macht der Prototyp, der eine zufällige Webseite als Authentifikationsmodul absichern kann. Um die Verfahren (MFA) noch mehr abzusichern, wäre es in Zukunft nötig die Adaptivität zu steigern, indem die Verfahren dem Nutzer nur präsentiert werden, falls das Userprofil und das vorhandene Betriebssystem / Gerät dies hergibt. Die Authentifikation gelingt den Menschen der Zukunft recht gut, sie muss dem Nutzer nur noch bequem gemacht und angeboten werden. So kann das Passwortproblem mit neuen Verfahren gelöst werden.


\cleardoublepage

\makedeclarationofauthorship % print the declaration of authorship
\printbibliography % Print the bibliography
\printnoidxglossary[type=\acronymtype] % print the list of abbreviations
\listoffigures % print the list of figures

\appendix % Begin the appendix. Chapters begin with letters
% \pagenumbering{Roman} %page numbers as capital roman numbers
\end{document}
% ----------------------------------------------------
