% !TeX root = ../my-thesis.tex
\chapter{Einleitung}


\section{Problemdefinition}
Lange reichte der Benutzername und das Passwort aus, um einen Benutzer sicher zu identifizieren. Heutzutage gelangen sensible Informationen immer öfter an Dritte. Dabei ist es nicht zwingend notwendig, dass die Authentifizierungsquelle, also jene Quelle bei der die Daten persistiert sind und die die Authentifizierung bei Eingabe durchführt, diese Daten irrtümlich herausgibt. Durch sogenannte Metadaten, das sind Informationen und Merkmale zu Daten, ist es Angreifern häufig möglich das Passwort zu erraten bzw. zurückzusetzen. Wenn also der Benutzername lautet 'MichaelJacksonFanForever' ist die Antwort auf die Sicherheitsfrage zum Lieblingskünstler nicht weit entfernt.
Gleichzeitig neigen Menschen zu leicht merkbaren Passwörtern, die sie mehrfach für verschiedene Dienste verwenden. Dies stützt eine kürzlich durchgeführte repräsentative Studie der Bitkom Reserach [1] im Auftrag des Digitalverbands Bitkom. So nutze etwa jeder dritte Onlinenutzer (36\%) dasselbe Passwort für mehrere Dienste. Auch wenn gleichzeitig 63\% der Befragten angaben, bei der Erstellung von Passwörtern auf ``einen Mix aus Buchstaben, Zahlen und Sonderzeichen'' zu achten, beweist diese Befragung an 1.000 Internetznutzern, dass die Frage nach der Sicherheit von Passwörtern auch im Jahre 2020 immernoch Relevanz hat. Eine ähnliche Studie hat die Bitkom zum Thema 'Nachlässigkeit bei Passwörtern' am 08.11.2016 [2] gemacht, bei der die prozentuale Verteilung an unsicheren Passwortnutzern die befragt wurden nur einen Prozent höher liegt. Das heißt konkret, dass sich innerhalb von 4 Jahren keine Besserung ergeben hat. Das Problem mit unsicheren Passwörtern ist allerdings so alt wie das Internet.

\section{Stand der Forschung}
\blindtext

\section{Zielsetzung}
\blindtext
