% !TeX root = ../my-thesis.tex
\chapter{Einleitung}

\section{Motivation}
Die Motivation für die Arbeit erschloss sich mir am 29. Juni 2020, als eine Rundmail von dem Monitoring-Tool von Firefox herumging, welches einen neuen Datenleck bei Wattpad meldete. Wattpad ist eine Webseite, bei welcher Nutzer Geschichten frei für die Öffentlichkeit schreiben und publizieren können. Kompromittiert wurden neben Passwörtern auch IP-Adressen, E-Mail-Adressen, Geburtsdaten und sehr persönliche Informationen wie geografische Standorte, ein Kurzprofil, das Profilbild, die Social-Media-Profile und alle verknüpften Accounts jeglicher Dienste des Benutzers. Die Datenmenge umschloss etwa 300 Millionen Einträge, bei denen die Passwörter zwar gesalted sind, allerdings das Knacken dieser bereits durch eine Hackercommunity in Angriff genommen wurde und etwa 10\% bereits 'entschlüsselt' wurden. Die Frage die sich stellte war es, ob persönliche Daten von mir nun im Umlauf waren. Glücklicherweise nutze ich einen Passwort-Manager und mein Passwort war zufallsgeneriert und ich kannte es selbst nicht einmal, da es zu kompliziert war um es sich zu merken. Dieses Leck hatte in meinem Freundeskreis, unter denen einige auf Wattpad registriert waren, rumgesprochen und mir stellte sich immer mehr eine Kernfrage: Wie kann es im Jahre 2020 immernoch sein, dass Menschen die selben Passwörter für Dienste verwenden und diese nicht ausreichend kompliziert wählen? Immerhin gibt es dieses Problem nun schon seit einigen Jahren und dennoch ändert sich augenscheinlich nichts an der Situation. Technische Probleme und Angriffsszenarien kriegt die informationsafine Menscheit gut organisiert, doch dann scheitert es in riesigen Unternehmen an der simplen Wahl eines Passwortes. Bei mehreren Disskusionen in meinem Umfeld, kamen die selben Argumente gegen eine zwei Faktor Authentifizierung und Passwort-Manager hervor. Passwörter seien unmöglich für alle Dienste merkbar und eine zwei Faktor Authentifizierung sei zu kompliziert um sie für jeden Dienst, wenn möglich, einzusetzen. Die Forschungsfrage kam daher, dass ich mich fragte, ob man mit einer Kombination aus den vorhandenen vielfältigen Authentifizierungsmöglichkeiten eine beqeueme aber gleichzeitig auch sichere Authentifizierungsvariante schaffen könnte, bei denen es dem Anwender kinderleicht möglich sein soll sich gegenüber einer Webseite zu authentifizieren. Das Merken von langen Passwörtern sollte mitunter vermieden werden, da diese bei Kompromittierung publik werden und danach ein potenzielles Risiko für jeden anderen Dienst darstellen.

\section{Problemdefinition}
Lange reichte der Benutzername und das Passwort aus, um einen Benutzer sicher zu identifizieren. Heutzutage gelangen sensible Informationen immer öfter an Dritte. Dabei ist es nicht zwingend notwendig, dass die Authentifizierungsquelle, also jene Quelle bei der die Daten persistiert sind und die die Authentifizierung bei Eingabe durchführt, diese Daten irrtümlich herausgibt. Durch sogenannte Metadaten, das sind Informationen und Merkmale zu Daten, ist es Angreifern häufig möglich das Passwort zu erraten bzw. zurückzusetzen. Wenn also der Benutzername lautet 'MichaelJacksonFanForever' ist die Antwort auf die Sicherheitsfrage zum Lieblingskünstler nicht weit entfernt.
Gleichzeitig neigen Menschen zu leicht merkbaren Passwörtern, die sie mehrfach für verschiedene Dienste verwenden. Dies stützt eine kürzlich durchgeführte repräsentative Studie der Bitkom Reserach [1] im Auftrag des Digitalverbands Bitkom. So nutze etwa jeder dritte Onlinenutzer (36\%) dasselbe Passwort für mehrere Dienste. Auch wenn gleichzeitig 63\% der Befragten angaben, bei der Erstellung von Passwörtern auf ``einen Mix aus Buchstaben, Zahlen und Sonderzeichen'' zu achten, beweist diese Befragung an 1.000 Internetznutzern, dass die Frage nach der Sicherheit von Passwörtern auch im Jahre 2020 immernoch Relevanz hat. Eine ähnliche Studie hat die Bitkom zum Thema 'Nachlässigkeit bei Passwörtern' am 08.11.2016 [2] gemacht, bei der die prozentuale Verteilung an unsicheren Passwortnutzern die befragt wurden nur einen Prozent höher liegt. Das heißt konkret, dass sich innerhalb von 4 Jahren keine Besserung ergeben hat. Das Problem mit unsicheren Passwörtern ist allerdings so alt wie das Internet.

\section{Stand der Forschung}
\blindtext

\section{Zielsetzung}
\blindtext
