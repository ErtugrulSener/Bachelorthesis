% !TeX root = ../my-thesis.tex
\chapter{Methodik}
\section{Prototypenaufbau}
Das Ziel des Prototypes ist es, wie eingangs erwähnt, vorhandene Authentifizierungsverfahren abseits der klassischen UserID / Passwort Methode zu begutachten und dessen Schwächen aufzudecken. Der Prototyp beschreibt eine einfache Webseite, die aus dem globalen Internet erreichbar sein wird und zwei Eingabefelder und einen Loginbutton besitzt. Die unterschiedlichen Methoden der Authentifizierung wählt man über ein Dropdownmenü über dem Login - Button. Je nach Authentifizierungsverfahren werden kleine Popup-Boxen sichtbar, die die weiteren Schritte für die Authentifikation erläutern. So muss bei der zwei Faktor Authentifizierung anhand von einem Fingerabdruck weder ein Username noch ein Passwort eingegeben werden. Die Webseite muss lediglich auf die Schnittstellen des Betriebssystems zugreifen, um den Nutzer zu authentisieren. Wie der Nutzer eingangs den Fingerabdruck eingerichtet hat, spielt für die Webseite keine Rolle.
Ein Teil des Prototypen soll die gewählten Methoden demonstrieren. Neben einer erfolgreichen Demonstration der Authentifizierung sollen nach jeder Methode auch Kennwerte ausgegeben werden. Einer davon soll zum Beispiel die Zeit von der ersten Eingabe in ein Eingabefeld bis zur Authentifizierung in Sekunden zählen und anzeigen. Weitere Messwerte sind denkbar und werden sich bei der Implementation ergeben. \\
\\
Neben den vorhandenen Methoden soll die eigene Architektur aufgebaut werden, die sich an vorhandenen Authentifikationsmöglichkeiten bedient. Die UserID und das Passwortfeld sind beim ersten Aufruf der Seite zwar zu sehen, müssen allerdings nicht zwingend für jedes der Verfahren genutzt werden, so kann es zum Beispiel bei einer besitzbasierten Authentifikation bereits reichen, den Besitz (z.B einen USB - Stick, welcher einen privaten Schlüssel beherbergt) im Computer einstecken zu haben und auf den Loginbutton zu drücken. Bei der Architektur muss zwingend eine Datenbank und ein Backend zur Webseite implementiert werden um einerseits die eigeggangenen Daten zu bearbeiten und andererseits in die Datenbank zu persistieren. Das Dropdownmenü zeigt die Authentifikation über biologischen Merkmale (Touch ID oder Face ID) nur sofern das Gerät, auf welchem der Prototyp bedient wir, einen entsprechenden Sensor und die entsprechende Software zur Verarbeitung besitzt.

\section{Auswahl der Authentifizierungsverfahren}
Neben des altbekannten \ac{totp} Verfahrens, wird das Secure Element und die E-Mail Authentifikation betrachtet. Dabei bedienen sich diese Verfahren aller drei Möglichkeiten der Authentifikation, dem Wissen, dem Besitz und der körperlichen Merkmale. 

\section{Kriterien zur Bewertung des Prototypen}
\section{Architektur}
