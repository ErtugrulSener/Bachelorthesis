% !TeX root = ../my-thesis.tex
\chapter{Grundlagen}

\section{Datenschutzkriterien}
\subsection{IT-Grundschutz}
Je nach Grundwert definiert der IT-Grundschutz welchen Schutzbedarf ein bestimmtes Asset oder ein Prozess besitzt. Allgemein kategorisiert werden die Schutzbedarfskategorien nach:
\begin{itemize} 
\item normal (Schadenauswirkungen begrenzt bis überschaubar)
\item hoch (Schadenauswirkungen könnten hoch bzw. beträchtlich sein)
\item sehr hoch (Schadensauswirkungen können ein existenziell bedrohliches Ausmaß annehmen
\end{itemize}
Bei dem Prototypen wird die allgemeine Schadensauswirkung wohl im Bereich normal liegen, da keine persönlichen Daten verarbeitet werden bzw. nur so viele Daten vom User benutzt werden, wie zwingend notwendig. (Nach dem Need-To-Know Prinzip) Sollte die Webseite oder Teile der Webseite publiziert bzw. im Business - Umfeld genutzt werden, muss eine Neubewertung der Daten nach Vertraulichkeit, Integrität und Verfügbarkeit stattfinden. Vor allem muss darauf geachtet sein, dass Daten über Fingerabdrücke verschlüsselt und Passwörter im gehashten Zustand in Datenbanken persistiert werden. Bei Kompromittierung des Hauptrechners, welches die größte Bedrohung in diesem Szenario darstellen würde, droht ein Data Breach mit dem Angreifer diese Daten weiterverwenden können. Durch Verschlüsselungen durch Schlüssel, die nicht auf dem Hauptsystem (oben u.a als Hauptrechner benannt) liegen. Gleichzeitig sollten die genutzten Verfahren insgesamt mathematisch sicher sein, auf veraltete Verschlüsselungsverfahren ist zu verzichten. Dies gillt auch für Hashes wie MD5, die mittlerweile durch sehr große vorgerechnete Tabellen erraten werden. \\\\
Im Falle einer Kompromittierung besäße der Schutzbedarf der Daten innerhalb der Datenbank, welche nicht gehasht oder anderweitig verschlüsselt sind, die Kategorie 'sehr hoch'. Eine Kompromittierung kann für das Unternehmen einen Imageschaden sowie weitreichende juristische Klagen zur Folge haben. Je nach dem wie kompliziert die Ursprungsdaten sind und welcher Hashingalgorithmus in Kombination mit Salt und Pepper genutzt wurde, können diese Daten an die Öffentlichkeit gelangen und Angreifer können die Passwörter für andere Dienste nutzen. Die Wahrscheinlichkeit für diesen Schaden ist noch vergleichbar niedrig, weshalb die Schadensauswirkung hoch statt sehr hoch ist. Beim Prototypen wird darauf geachtet werden, dass die Antwort zu möglichen Sicherheitsfragen auch verschlüsselt bzw. gehasht gespeichert werden und die Frage nicht in der Datenbank im Klartext steht sondern nur eine Zahl ist, die von der Webseite interpretiert wird und nie öffentlich gezeigt wird. \\
Der IT-Grundschutz definiert drei Arten der Absicherung. Die Basis-Absicherung ist relevant für Institutionen die einen Einstieg in den IT-Grundschutz suchen und relativ schnell alle relevanten Geschäftspozesse mit einfach umzusetzenden Basismaßnahmen sichern wollen. Die Kern-Absicherung konzentriert sich auf besonders wichtige Geschäftsprozesse und vertieft sich in die Sicherung dieser. Von einer Standard-Absicherung spricht man, wenn alle empfohlenen IT-Grundschutz-Vorgehensweisen durchgeführt werden. Sie beschreibt den allumfassenden Schutz der Prozesse und Bereiche der Institution. [9]

\subsection{ISO 27001}
Zunächst einmal muss unterschieden werden zwischen dem Standards ISO 27001 und der Zertifizierung von ISO 27001 auf Basis des IT - Grundschutzes. Im Grundansatz sind diese beiden Ansätze soweit kompatibel, da sie beide ein \ac{isms} beschreiben, um Risiken in der Informationssicherheit zu dezimieren (oder im besten Fall zu eliminieren). Gleichzeitig beschreiben beide eine Kontinuität in der regelmäßigen Überprüfung und Neubewertung dieser Risiken. Ein wesentlicher Unterschied zwischen den beiden Standards liegt in der Risikoanalyse. Während beim BSI-Grundschutz eine Risikoanalyse nur in besonderen Fällen erforderlich ist, ist sie beim Standard ISO 27001 ein fester Bestandteil, an das keine Bedingungen geknüpft sind. Wobei streng genommen häufig der eigentliche Schriftzug zur Risikoanalyse in der Norm ISO 27005 steht und darauf häufig verwiesen wird. Gleichzeitig sei laut Dr. Markus a Campo der Standard ISO 27001 mehr an dem Management der Informationssicherheit interessiert, wogegen der BSI-Grundschutz sich auf die detaillierte Vorgehensweise zur Minimierung von Risiken stütze. Unabhängig vom gewählten Standard sei es außerdem sinnvoll, die eigene Sicherheit in regelmäßigen Abständen zu überprüfen.
Laut dem \ac{bsi} ist die ISO 27001 Zertifizierung auf Basis des IT-Grundschutzes für sowohl Standard als auch die Kern-Absicherung möglich. [8] Die Zertifizierung auf Basis des IT-Grundschutzes erfüllt

\section{Authentifizierungsmethoden}
\subsection{Klassische Passwörter}
\subsection{Einmalkennwörter}
\subsection{Authentifizierung über Schlüssel}
\subsection{Biometrisierte Authentifizerung}
