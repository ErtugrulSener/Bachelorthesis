% !TeX root = ../Bachelorarbeit.tex
\chapter{Ausblick und Fazit}
Die Userauthentifikation ist eine der größten Herausforderungen an Webanwendungen der Neuzeit und wird es voraussichtlich auch bleiben. Das Verfahren der Negativauthentifikation zeigt in welche Richtung zukünftige Verfahren wohl gehen: Man versucht die alteingesessenen Passwörter durch neue Denkansätze und Verfahren besser zu schützen. In diesem konkreten Beispiel macht man wortwörtlich das Gegenteil einer gewöhnlichen Authentifikation und versucht dadurch falsche Anfragen zu erkennen bevor der Server sie überhaupt prüfen muss. Zusätzlich scheint der Zufall in den Fokus zu kehren, während die gewöhnliche Multi-Fator-Authentifikation feste Auswahlmöglichkeiten bot, ist die A-MFA darauf spezialisiert dem User seinen Parametern zugeschnittene Möglichkeiten zu bieten. Das macht es Angreifern ungemein schwer, einen bestimmten Nutzer konkret anzugreifen - Da jeder Nutzer ein potenziell anderes Preset bzw. 'Profil' besitzen wird, das seine Authentifizierungsmöglichkeiten ergibt. Neben neuen Ansätzen muss sich die zukünftige Forschung wohl vermehrt mit der Haltung der Nutzer zum Thema Bequemlichkeit widmen. Neuere Verfahren, die aus technischer Sicht sehr sicher sind, können nicht bestehen wenn sie vom Nutzer nicht als angenehm bzw. bequem empfunden werden. Die Digitalisierung zwingt Webseitenbetreiber (durch öffentlichen Druck und Konkurrenz) dazu, alle vorhandenen Möglichkeiten zur Authentifikation anzubieten und dem Nutzer die Wahl dessen zu lassen. Das Selbe macht der Prototyp, der eine zufällige Webseite als Authentifikationsmodul absichern kann. Die Authentifikation gelingt den Menschen der Zukunft recht gut, sie muss dem Nutzer nur noch bequem gemacht und angeboten werden. So kann das Passwortproblem mit neuen Verfahren gelöst werden.
